% Admir
\section{Types of mobile sensing}  
% TODO: 
% Apply definitions
% Explain difference between participatory and opportunistic sensing
% Relate to articles.
% Relate to app (start geofence -> continuous sensing -> end geofence)
% Reflect and discuss. 


%**************************************Definition Software 
\begin{defi}[\textbf{Participatory Sensing}]
Participatory  Sensing  emphasizes  the  involvement  of  citizens  and  community  groups  in  the 
process  of  sensing  and  documenting  life  where  they  live,  work,  and  play. \cite{Goldman2009}. 
\end{defi}
% https://www.researchgate.net/publication/42831772_Participatory_Sensing_A_Citizen-Powered_Approach_to_Illuminating_the_Patterns_that_Shape_our_World

\begin{defi}[\textbf{Opportunistic Sensing}]
Opportunistic sensing shifts the burden of supporting an application from the custodian to the sensing system, automatically determining when devices can be used to meet application requests. \cite{Lane:2008:USS:1411759.1411763} 
\end{defi}



\begin{enumerate}
    \item  Participatory sensing
    
    \begin{enumerate}
        \item  User actively engages in the data collection activity

        \item  Manually determines how, when, what, and where to sample
   
    \end{enumerate}
            
    \item  Opportunistic sensing
    
        \begin{enumerate}
        \item   Data collection runs as a continuous background activity

    \end{enumerate}
\end{enumerate}

\section{Application Types}
% TODO: 
% Apply definitions
% Explain difference between indivudual, group and community
% Relate to articles.
% Relate to app 
%   - App is individual (now)
% Reflect and discuss. 
%   - Group could be used for sensing (location)
%     precision/accurarcy (future)
%   - Community could be to provide route data for 
%     municipality (future)
    \begin{enumerate}
        \item  Individual activity sensing fitness applications, behavioural suggestions

        \item  Group activity sensing groups to sense common activities and help achieving group goals. Eg: assess neighbourhood safety, collective recycling efforts 
        
        \item Community sensing large scale sensing, where a large number of people have the same application installed. E.g., tracking spread of disease across a city, congestion in a city.
    \end{enumerate}
    
\section{Context Awareness}
% TODO:
% Apply definitions
% Explain context awareness, classification and phone context
% Give example of how to combine multiple sensors
%   - ex: gyroscope and accelerometer
% Relate to articles
% Relate to app: how we use classification
% Reflect and discuss. Can context awareness be used in other ways in our app?

% Relate to activity recognition (uses accelerometer

